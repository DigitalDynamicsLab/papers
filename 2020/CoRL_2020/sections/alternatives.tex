%----------------------------------
The big players in the ``V'' field: MSC.ADAMS of MSC.Software \cite{ADAMSsoftware,adamsManual}, SimPack of Dassault Syst\'emes \cite{SimpackSoftware}, RecurDyn of FunctionBay \cite{functionBay}, and Siemens' Virtual.Lab \cite{lmsVirtualLab}, do not have any product that covers the A or C axes. Limited support for virtual worlds along the ``E'' exists, yet it is aligned with manufacturing applications for automation purposes. These softwares are expensive and not open source.

Simulation tools are available for virtual testing and validation of Advanced Driver-Assistance Systems (ADAS).  Most of them span the A and E axes well. Connectivity is seldom considered, and the support along the V-axis relies on rudimentary vehicle models or third party solutions.  With the notable exception of Gazebo~\cite{gazebo}, these are mostly commercial, closed-source solutions. As such, they cannot be easily extended to test new vehicle models, tinker with virtual sensing, or augment traffic scenarios to include, for instance, heterogeneous agents (pedestrians and riders).

%----------------------------------
Gazebo \cite{gazebo,gazebo-VRC} (collaboration letter included) is an Apache 2.0 license, open-source ``3D dynamic simulator with the ability to accurately and efficiently simulate populations of robots in complex indoor and outdoor environments. While similar to game engines, Gazebo offers physics simulation at a much higher degree of fidelity, a suite of sensors, and interfaces for both users and programs'' \cite{gazebo, gazebo-VRC}. While Gazebo has been successfully used for DARPA's Virtual Robotics Challenge~\cite{gazebo-VRC}, its V-axis relies on physics engines not suited, or at least not equipped, to model ground vehicles at the level of sophistication and fidelity required by a CAVE. Lastly, for reasons discussed in section \S\ref{ss:preamble-A-axis}, its sensing prowess is lacking. 

%----------------------------------
The Virtual Autonomous Navigation Environment (VANE), developed by the US Army Engineer Research and Development Center is a ``high-fidelity, physics-based software that simulates terrain and environmental conditions.'' It is intended to support the design, development, and testing of autonomy algorithms for intelligent ground vehicles \cite{VANE-2012}. It consists of a set of independent models for sensors, vehicle dynamics, and terramechanics. The VANE uses high-fidelity environment geometry and ray-tracing to simulate GPS, LiDAR, and camera sensors. Inertial sensors are simulated using the dynamics engine. Closed-source and unavailable to outside researchers, VANE also relies on a physics engine, the Open Dynamics Engine (ODE)~\cite{ODE-website, Drumwright-2010} originating in the video gaming industry. Current VANE development efforts, unrelated to this group, aim at interfacing to {\CHRONO}::Vehicle.

%----------------------------------
The Autonomous Navigation and Virtual Environment Laboratory (ANVEL) is a real time simulation software desktop application developed by Quantum Signals that ``enables the analysis and testing of autonomous vehicle dynamics, terrain/obstacle interaction, sensors, algorithms, and operation in a virtual world'' ~\cite{ANVEL-website}. ANVEL serves as a real-time, desktop interface for the VANE. As of June 2013, ANVEL is available to qualified academic institutions. Like VANE, ANVEL relies on the game physics engine ODE.

%----------------------------------
PreScan~\cite{PreScan-website}, developed by TASS International, is a simulation platform for advanced driver assistance systems and uses a MATLAB/Simulink interface. It can be used for model-based controller design, software-in-the-loop, and hardware-in-the-loop (HIL) systems. PreScan can operate in open-loop and closed-loop, offline and online modes. It is an open software platform which has flexible interfaces to link to third party vehicle dynamics commercial solutions such as CarSIM and dSPACE ASM, and third party HIL simulators/hardware, e.g., ETAS, dSPACE, and Vector. It has no support along the C and A axes.

%----------------------------------
Pro-SiVIC~\cite{Hiblot-2010, Belbachir-2015}, developed by the French Institute of Science, Transport Technology and Network (IFSTTAR) and commercialized by CIVITEC (a spin-off of IFSTTAR), provides very good support along the A and E axes, with no support for the C and V axes. It is widely used for Advanced Driver Assistance Systems (ADAS), as it provides a family of sensors in a virtual environment. 

%----------------------------------
Finally, there are gaming engines, e.g., \cite{unityGaming,unrealEngine}, which may or may not be open source. Using the nomenclature adopted in this document, they support avatar vehicles. There is no support for sensing or connectivity insofar simulating DSRC is concerned. Some game engines provide support along the E-axis, yet there is little concern with the interplay between sensing and the environment. Along the V-axis, the underlying dynamics is of little concern as long as the motion looks realistic.


