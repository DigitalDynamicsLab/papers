\documentclass[11pt]{nodycon2021}
\usepackage{epsfig,amsmath,amsfonts}

%%
\let\OLDthebibliography\thebibliography
\renewcommand\thebibliography[1]{
  \OLDthebibliography{#1}
  \setlength{\parskip}{0pt}
  \setlength{\itemsep}{0pt plus 0.3ex}
}
\usepackage[official]{eurosym}
%%%

\title{Template for NODYCON 2021 One-page abstract}

\author{\textbf{First Author}$^\ast$, Second Author$^\ast$ and Third Author $^{\ast\ast}$}

\address{$^\ast$Department of Mechanical and Aerospace Engineering, Florida Institute of Technology, Melbourne, FL, USA\\	 $^{\ast\ast}$Institution, City, State, Country}

\abstract{The following guidelines have been prepared for authors of papers to be presented at the \textit{Second International Nonlinear Dynamics Conference NODYCON 2021} originally planned to be held at Sapienza University of Rome, in Rome (Italy), February 16-19, 2021, and now hosted as a virtual online conference. These guidelines provide the rules for the preparation of abstracts (maximum 1 page). Authors are invited to follow these guidelines for inclusion of their abstracts in the digital Book of Abstracts.}
\begin{document}
%%%%%%%%%%%%%%%%%%%%%%%%%%%%%%%%%%%%%%%%%%%%%%%%%%%%%%%%%%%%
\section{Introduction}

The full scope of the conference which reflects the rich spectrum of topics covered by NONLINEAR DYNAMICS can be found in the Call for Papers available on the conference website https://nodycon.org/. The one-page abstracts to be submitted online using the Conference portal    will be included in the Conference Book of Abstracts. The deadline for submitting the one-page Abstract is Friday, July 10, 2020.  Authors will have the option of submitting full-length papers for publication consideration as Springer Conference Proceedings. The papers will be made available to the following indexing services: Conference Proceedings Citation Index (CPCI), part of Clarivate Analytics’ Web of Science, EI Engineering Index (Compendex and Inspec databases), Google Scholar, MathSciNet, Scopus, Zentralblatt MATH. A short list of papers will be invited to a Special Issue of NONLINEAR DYNAMICS. To accelerate the reviewing process, referees will be invited from those who will attend the conference. The Springer Conference Proceedings papers will be possibly made available during NODYCON 2021. The website will be open for full-length papers submission until September 11, 2020.  The papers must be written in English and should present material that is novel and  unpublished at the time of the Conference. Prospective presenters are asked to submit their abstracts through the Conference website. The electronic submission of the one-page Abstract is required in PDF format only. The template for the manuscript preparation according to the conference guidelines can be downloaded both in Word Office and LaTeX formats. The layout of the one-page Abstract has to follow the style of this document, starting with the title, followed by the author(s) names and their affiliation(s). \textbf{The name of the author presenting the paper should be highlighted in boldface}.

\begin{figure}[h!]
\centering
\epsfig{file=BannerNodycon2021.eps,scale=0.75}
  \caption{Welcome to the Second International Nonlinear Dynamics Conference NODYCON 2021.}
\label{fig:Fig1}
\end{figure}


\section{Electronic submission of the one-page Abstract (PDF format)}
The one-page  Abstract should be organized in non-numbered sections  by adopting the format used in these guidelines. In particular, the headings have to be centred in bold lower-case letters (12 pt). The text should be single-spaced using 11 pt font. The typing area of the A4 page should be 170 x 267 mm, leaving 20 mm left and right margins, and 15 mm top and bottom margins. The total length of the  Abstract, including one figure and the references, should be one page. The text should be written using Times New Roman font. The name of the author presenting the paper should be highlighted in boldface. The title should appear 15 mm below the top edge of the page. The manuscript has to be organized as follows: i) a brief, descriptive \textit{Abstract} of approximately 150 words; ii) a short \textit{Introduction} to the problem formulation containing also the motivation and the purpose of the study together with a brief summary of previous works on the topic; iii) \textit{Results and discussion} containing the core  findings and highlights  of the work. Finally, the \textit{References} list should be collocated at the end of the page. 

\begin{thebibliography}{99}

\bibitem{Cooler} Cooler A. S. (1999) Binary Flow Systems.  {\em J. Fluid Mech}  {\bf 999}:999-996.

\bibitem{Icer}	Icer D.F., Adams J.A. (1977) Mathematical Elements for Computer Simulation. McGraw Hill, NY.

\bibitem{NODYCON}	NODYCON 2021, website \textbf{https://nodycon.org/}.

\end{thebibliography}
 \end{document} 