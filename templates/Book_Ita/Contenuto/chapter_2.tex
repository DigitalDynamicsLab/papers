
\chapter{Formula di Lagrange}
\label{Formula di Lagrange} 
%\textbf{Nota:} Per facilitare la lettura in seguito si userà il grassetto per le quantità vettoriali mentre l'overline indicherà che le coordinate in questione sono relative. \newline
\section{Lavoro Virtuale}
Un passaggio essenziale nella formulazione lagrangiana delle equazioni della dinamica del sistema multibody è il calcolo delle forze generalizzate associate con le forze generalizzate del sistema.
In questa sezione le forze generalizzate sono introdotte applicando il principio dei lavori virtuali nei casi statico e dinamico. Nello svolgimento in seguito si considera un sistema di particelle.
Assumendo che un corpo rigido non è altro che un grande numero di particelle queste espressioni saranno estese ai corpi rigidi con poche modifiche.
\paragraph{Equilibrio Statico}
Si consideri un sistema composto da $n_p$ particelle. Se il sistema di particelle è in equilibrio significa che 
\begin{equation}
\label{eq:equilibrium_virtualwork}
\mathbf{ \sum_{i=1}^{n_p} F^i \delta r^i }=  0
\end{equation}
Dove F ed r sono la forza agente e lo spostamento virtuale relativi ad un particella i. La forze sono di 2 tipi: forze esterne e forze di vincolo: 
$F^i_e +F^i_c = F^i$ ed applicando questa suddivisione all'equazione \ref{eq:equilibrium_virtualwork} si ottiene:
\[\sum_{i=1}^{n_p}F^i\delta r^i = 0 \Rightarrow \sum_{i=1}^{n_p}F_e^i\delta r^i + \sum_{i=1}^{n_p}F_c^i\delta r^i = 0 \Rightarrow \delta W_e + \delta W_c = 0 \]
Lo spostamento di ogni particella è funzione delle coordinate del sistema $q= (q_1, q_2,\quad ... \quad q_n)$.
\[r_i = r_i(q_1, q_2,\quad ... \quad q_n) \Rightarrow \delta r_i = 
\frac{\partial r_i}{\partial q_1}\delta q_1 + \frac{\partial r_i}{\partial q_2}\delta q_2 + \frac{\partial r_i}{\partial q_3}\delta q_3 + .... + \frac{\partial r_i}{\partial q_n}\delta q_n \]
Si consideri il caso in cui i vincoli non compiono lavoro (\emph{workless constraints}), ad esempio quando i vincoli non hanno attrito (o la forza di attrito viene inserita fra le forze esterne). In questo caso il lavoro virtuale è dovuto alle sole forze esterne: \begin{equation}
\delta W = \delta W _e = 
\sum_{i=1}^{n_p}\left( F_e^i \sum_{j=1}^n \frac{\partial r_i}{\partial q_j}\delta q_j \right) = 0
\end{equation}
Si definisce $Q_j$ come: \begin{equation}
Q_j = \sum_{i=1}^n F_e^i \frac{\partial r_i}{\partial q_j} = \sum_{i=1}^n \left(F_e^i\right)^T r^i_{q_j} \qquad \left( r^i_{q_j}=\frac{\partial r_i}{\partial q_j} \right)
\end{equation}
Per cui si ottiene: \begin{equation}
\delta W = \delta W_e = \sum_{j=1}^n Q_j\delta q_j = Q^T \delta q = 0
\end{equation}
Con \begin{equation}
Q = \left[Q_1, Q_2 , .... , Q_n\right]^T
\end{equation}
Vettore delle forze generalizzate.

\paragraph{Equilibrio dinamico, Principio di D'Alembert}
La seconda legge di Newton afferma che: \begin{equation}
\label{eq:newton_law}
F^i = \dot{P}^i \qquad o \qquad F^i - \dot{P}^i = 0
\end{equation}
Con F risultante delle forze e P quantità di moto agenti sulla generica particella i. 
Da \ref{eq:newton_law} esplicitando il lavoro virtuale di un sistema di particelle in equilibrio dinamico si 
ottiene il \emph{principio di D'Alembert}
\begin{gather} 
\label{eq:dalembert_principle}
\sum_{i=1}^{n_p} (F_e^i + F_c^i -\dot{P}^i )\delta r^i = \sum_{i=1}^{n_p} (F_e^i -\dot{P}^i )\delta r^i + \sum_{i=1}^{n_p} F_c^i\delta r^i = 0 \nonumber \\ \underset{workless constraints}{\Rightarrow} \sum_{i=1}^{n_p} (F_e^i -\dot{P}^i )\delta r^i = 0 \nonumber \\ \Rightarrow \sum_{i=1}^{n_p} (F_e^i -\dot{P}^i )\sum_{j=1}^n \frac{\partial r_i}{\partial q_j} \delta q_j = 0
\end{gather}

Ridefinendo $Q_j$ nel caso dinamico comprendendo il contributo delle forze d'inerzia si ottiene:
\begin{equation}
Q_j = \sum_{i=1}^{n_p} \left(F_e^i-\dot{P}^i\right)\frac{\partial r^i}{\partial q_j}
\end{equation}
Da cui:
\begin{equation}
\sum_{j=1}^n \sum_{i=1}^{n_p} \left(F_e^i-\dot{P}^i\right)\frac{\partial r^i}{\partial q_j}\delta q_j = \sum_{j=1}^n Q_j \delta q_j = Q^T = 0
\end{equation}
\section{Dinamica Lagrangiana per particelle}
Si introduce qui la formulazione lagrangiana della dinamica di un sistema di particelle che verrà in seguito estesa
ai corpi rigidi.
\subsection{Lavoro Virtuale su un sistema di particelle}
\paragraph{Spostamento di una particella \emph{i}}
Lo spostamento di una particella \emph{r} è funzione del sistema di coordinate \emph{q} e del tempo \emph{t}:
\begin{equation}
r^i = r^i\left(q_1, q_2, q_3 ... ,q_n, t\right)
\end{equation}
Da cui si ottiene, derivando a catena:
\begin{align}
\label{eq:particle_disp_time_der}
\dot{r}^i = \frac{D r^i}{D t} & = \frac{\partial r^i}{\partial q_1}\dot{q_1} + \frac{\partial r^i}{\partial q_2}\dot{q_2}
+ ... + \frac{\partial r^i}{\partial q_n}\dot{q_n} + \frac{\partial r^i}{\partial t} \nonumber \\
& = \sum_{j=1}^n \frac{\partial r^i}{\partial q_j}\dot{q_j} + \frac{\partial r^i}{\partial t}
\end{align}
Si osservi che la sommatoria è dovuta alla variazione delle coordinate della particella nel riferimento, mentre il secondo termine è la variazione di $r^i$ mantenendo $q_j$ fissato, per cui potremmo vedere i due termini come una velocità relativa ed una velocità di trascinamento di un punto rispetto ad un sistema di riferimento non inerziale. \newline
Lo spostamento virtuale si può esprimere in funzione delle coordinate $q_j$ come:
\begin{equation}
\label{eq:particle_virtual_disp}
\delta r^i = \sum_{j=1}^n\frac{\partial r^i}{\partial q_j}\delta q_j
\end{equation}
\paragraph{Lavoro su una particella \textit{i}}
Dalla formula \ref{eq:particle_virtual_disp} per lo spostamento virtuale della particella si ottiene il lavoro virtuale della forza agente sulla particella 1\emph{i}:
\begin{equation}
\label{eq:virt_work_particle}
{F^i}^T \delta r^i = \sum_{j=1}^n {F^i}^T \frac{\partial r^i}{\partial q_j}\delta q_j
\end{equation}
Estendendo l'equazione \ref{eq:virt_work_particle} al sistema di particelle:
\begin{equation}
\label{eq:virt_work_particlesyst}
\sum_{i=1}^{n_p} {F^i}^T \delta r^i = \sum_{i=1}^{n_p}\sum_{j=1}^n {F^i}^T \frac{\partial r^i}{\partial q_j}\delta q_j = \sum_{j=1}^n Q_j \delta q_j
\end{equation}
Dove $Q_j$ è chiamato le \emph{componente delle forze generalizzate associato alla coordinata $q_j$} ed è:
\begin{equation}
Q_j = \sum_{i=1}^{n_p} {F^i}^T \frac{\partial r^i}{\partial q_j}
\end{equation}
\paragraph{Lavoro virtuale della forza di inerzia}
Il lavoro virtuale di tutte le forze d'inerzia del sistema può essere scritto come somma dei lavori virtuali delle forze di inerzia agenti di ogni \emph{i}-esima particella di massa $m^i$:
\begin{equation}
\label{eq:inertia_vwork_particle}
\delta W_i = \sum_{i=1}^{n_p}m^i\ddot{r}^i\cdot \frac{\partial r^i}{\partial q_j}\delta q_j 
\end{equation}
Un' utile identità:
\[\sum_{i=1}^{n_p}\frac{d}{dt}\left(m^i\dot{r}^i\frac{\partial r^i}{\partial q_j}\right)
 = \sum_{i=1}^{n_p} \left(m^i\ddot{r}^i\frac{\partial r^i}{\partial q_j}\right) +  \sum_{i=1}^{n_p} \frac{d}{dt} \left(m^i\dot{r}^i\frac{d}{dt}\frac{\partial r^i}{\partial q_j}\right) \]
Da cui:
\begin{equation}
\label{eq:part_inertia_work_identity}
\sum_{i=1}^{n_p}\left(m^i\ddot{r}^i\frac{\partial r^i}{\partial q_j}\right) 
 = \sum_{i=1}^{n_p}\left[ \frac{d}{dt}\left(m^i\dot{r}^i\frac{\partial r^i}{\partial q_j}\right) - m^i\dot{r}^i\frac{d}{dt}\left(\frac{\partial r^i}{\partial q_j}\right) \right]
\end{equation}
Si ottiene grazie all'equazione \ref{eq:particle_disp_time_der}
\begin{equation}
\label{eq:drdot_dq}
\frac{d}{dt}\left(\frac{\partial r^i}{\partial q_j}\right) = \sum_{k=1}^n \frac{\partial ^2 r^i}{\partial q_j\partial q_k}\delta \dot{q}_k + \frac{\partial ^2r^i}{\partial q_j\partial t} = \frac{\partial \dot{r}^i}{\partial q_j}
\end{equation}
Inoltre, sempre a partire dalla \ref{eq:particle_disp_time_der} e considerando la derivata parziale di $\dot{r}^i$ rispetto ad $q_j$:
\begin{equation}
\label{eq:drdot_dqdot}
\frac{\partial\dot{r}^i}{\partial \dot{q}_j} = \frac{\partial r^i}{\partial q_j}
\end{equation}
Dall'identità \ref{eq:part_inertia_work_identity} si ottiene:
\begin{align}
& \sum_{i=1}^{n_p} \frac{d}{dt} \left(m^i\ddot{r}^i\frac{\partial r^i}{\partial q_j}\right) 
 = \sum_{i=1}^{n_p}\left[ \frac{d}{dt}\left(m^i\dot{r}^i\frac{\partial r^i}{\partial q_j}\right) - m^i\dot{r}^i\frac{d}{dt}\left(\frac{\partial r^i}{\partial q_j}\right) \right]  \nonumber \\
&= \sum_{i=1}^{n_p}\left\{\frac{d}{dt}\left[\frac{\partial}{\partial \dot{q}_j}\left(\frac{1}{2}m^i\dot{r^i}^T\dot{r}^i\right)\right] -\frac{\partial}{\partial q_j}\left(\frac{1}{2}m^i\dot{r^i}^T\dot{r}^i\right)\right\}
\end{align}
\subsection{Equazione di Lagrange}
Denotando con $T^i$ l'energia cinetica della \emph{i}-esima particella:
\begin{equation}
\label{eq:part_kinenergy}
T^i = \frac{1}{2}m^i\dot{r^i}^T\dot{r}^i
\end{equation}
E con \emph{T} l'energia totale del sistema:
\[T = \sum_{i=1}^{n_p}t^i\]
Si ottiene:
\begin{align}
\label{eq:part_inertia_j}
\sum_{i=1}^{n_p} \left(m^i\ddot{r}^i\frac{\partial r^i}{\partial q_j}\right) 
&= \sum_{i=1}^{n_p}\left\{\frac{d}{dt}\left[\frac{\partial}{\partial\dot{q}_j}(T^i)\right] -\frac{\partial T^i}{\partial q_j} \right\} \\
&= \frac{d}{dt}\left(\frac{\partial T}{\partial\dot{q}_j}\right) - \frac{\partial T}{\partial q_j} 
\end{align}
Sostituendo \ref{eq:part_inertia_j} in \ref{eq:inertia_vwork_particle} ed applicandola al principio di D'Alembert (\ref{eq:dalembert_principle}
si perviene alla \emph{Formula di Lagrange} :
\begin{equation}
\label{eq:lagrange_notind}
\sum_j\left[ \frac{d}{dt}\left( \frac{\partial T}{\partial\dot{q}_j}\right) -\frac{\partial T}{\partial q_j} -Q_j \right] = 0
\end{equation}
Se le coordinate $q_j$ sono linearmente indipendenti si ha:
\begin{equation}
\label{eq:lagrange}
\frac{d}{dt}\left( \frac{\partial T}{\partial\dot{q}_j}\right) -\frac{\partial T}{\partial q_j} -Q_j = 0 \qquad j = 1, 2, ... ,n
\end{equation}
Si può riscrivere la \ref{eq:lagrange_notind} (e conseguentemente la \ref{eq:lagrange}) in forma matriciale:
\begin{equation} \label{eq:lagrange_matrix}
\left[ \frac{d}{dt} \left(\frac{\partial T}{\partial \dot{q}} \right) - \frac{\partial T}{\partial q} - Q_e^T \right] \delta q = 0
\end{equation}
%
